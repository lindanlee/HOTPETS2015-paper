\documentclass{sig-alternate-hotpets15}
\usepackage{color}
\usepackage[hyphens]{url}
\usepackage{longtable}
\usepackage{graphicx}
\usepackage{enumitem}
\usepackage{pdfpages}

\def\etal{{\it et al.~}}
\newenvironment{packed_enum}{
\begin{enumerate}
  \setlength{\itemsep}{1pt}
  \setlength{\parskip}{0pt}
  \setlength{\parsep}{0pt}
}{\end{enumerate}}
\newenvironment{packed_item}{
\begin{itemize}
  \setlength{\itemsep}{1pt}
  \setlength{\parskip}{0pt}
  \setlength{\parsep}{0pt}
}{\end{itemize}}

\begin{document}

\title{Assessing Usability of Tor Browser}
\numberofauthors{1}
\author{
 \alignauthor David Fifield\textsuperscript{1} and Linda N. Lee\textsuperscript{1}, Serge Egelman\textsuperscript{1,2}, David Wagner\textsuperscript{1}\\
   \vspace{0.5em}
   \affaddr{\textsuperscript{1}University of California, Berkeley, \{fifield,lnl,egelman,daw\}@cs.berkeley.edu}, \\
   \affaddr{\textsuperscript{2}International Computer Science Institute, egelman@icsi.berkeley.edu}\\
}

%\numberofauthors{1}
%\author{
% \alignauthor Anonymous\\
%   \vspace{0.5em}
%   \affaddr{Some Place}
%}
\maketitle

\begin{abstract}
Our talk will be about ongoing experiments aimed at
assessing and improving the usability of Tor Browser.
There have been been no major usability evaluations of
Tor Browser since the introduction of the 4.0 series,
which introduced radical UI changes.
We will briefly describe the results of a completed study
that examined downloading and installing the browser
and using it for basic browsing tasks.
This study uncovered a number of bugs and stopping points
and has already effected concrete change in the browser.
We will describe our plans for future studies aimed at
isolating specific browser features in depth,
with a detailed design of an experiment to evaluate
Tor Browser as a tool for censorship circumvention.
{\color {red}LL: I don't know how to say this, but I want the fact that we are evaluating it from a different standpoint --ie as a censorship circumvention tool, to stand out more. Also, we spend more abstract real estate talking about the first experiment rather than our design, which I think should be flipped. But for right now, I can't think of what exactly to fix so I'll look at this later.}
We solicit feedback on our experimental design
and the specific features of the browser that we are testing.
\end{abstract}

\keywords{Privacy, Security, User Studies, Anonymity, Tor} % NOT required for Proceedings

\section{What we already did/background stuff}
Tor is an anonymity tool which is used by a lot of people. It is important (prove this/cite stuff/say something more meaningful about this) and works well. However, there are complaints that Tor is not usable. Norcie did an experiment \cite{norcie2012eliminating} which identified stopping points in Tor browser, finding out when people would get frustrated with Tor so much they would stop using it. But what about the people who are using Tor? what usability issues would thee be if people were not stopped? 

To our knowledge, that has been the only user study of Tor. Since then, Tor has a lot of updates. Additionally, there were a lot of features that were untested. We ran a small pilot study on 5 journalists which involved a cognitive walkthrough.(explain cognitive walkthrough if necessary). We did this to find new stopping points, but also to see how they would download Tor, if they could understand the address bar/menu, how well they were able to complete basic tasks (searching, setting new circuits, etc), and generally if they had any usability complaints about the browser. 

Recruitment here. Demographics here. Setup here. Explain that other Tor users could see this as well. How long each the experiment took on average here. (put any misc details or additional methodology things here). The result was that people did have difficulty with X, Y, Z.

\section{Design}
After our first experiment, it was clear to us that so many of the features had been left uninvestigated--such as advanced web browsing tasks, the configuration menu, (what else were we going to investigate?), etc. But rather than selecting the features to study in isolation, we decided to study the new context in which the Tor browser is being used, censorship circumvention. To our knowledge, this is the first user study investigating the usability of Tor as a censorship circumvention tool, rather than an anonymity tool. 

For users to circumvent censorship in their resident countries, they will need to configure Tor to set up a proxy, bridge, or both. There is a configuration wizard to help guide users through setting up their connection, but our hypothesis is that the average user will not easily be able to configure Tor correctly in these adverse settings, as they require the user to provide IP addresses of proxies or bridges to use. The goal of our experiment is to see how successful users are at carrying out common browsing tasks in an adversarial setting using Tor. 

We start off the experiment by telling them that they are in an adversarial setting, and that some websites are blocked and some websites are not. We will instruct them to visit notblocked.com and also blocked.com to illustrate the situation that they are in. We will also explain what Tor is, and how they can use it if necessary {\color {red} is this true?.. or what happens?} We will them instruct them to complete the set of tasks. We designed the experiment so that the users are able to complete some easy tasks without having to use a censorship circumvention tool. To complete all the tasks, the participants will ultimately need to get the correct configuration settings for the country they are simulated to be in. We will end the experiment with a survey which asks them things like their security background, tech exposure, what was unexpected, what was hard, etc.

{\color {red} might be too much}We plan to recruit at least 120 (to 200?) users for the purpose of this study. This study will be performed at the Experimental Social Science Laboratory (Xlab) at the University of California, Berkeley. The laboratory space consists of 36 Toshiba Tecra R850 laptops in 6 rows of 6 laptops each, with cubicle walls separating each laptop. We used individual host firewalls to simulate the censorship environments. We plan to write a browser extension to measure when users visit what websites in order to log which websites they were successfully able to visit. The total length of the experiment should be less than an hour. Participants will be compensated \$30 for their time, which is more to cover minimum wage and transportation costs.  

We plan to analyze: which configuration tasks are hard (bridges vs proxies, etc.), how many were successful, how long it took users to configure, if they understood the process, if they were confident that it was working as expected, etc. 

We plan to simulate three censorship environments.
They are informed by our experience with pluggable transports
and knowledge of commonly seen censorship techniques.
They are not meant perfectly to replicate the network environment
in any particular country; rather, they are abstract simulations
that are nevertheless inspired by reality.

\begin{itemize}
\item{Corporate network.}
A simulation of a corporate or educational firewall.
Blocks all services but HTTP, HTTPS, and DNS.
Certain non-work-related domains, like youtube.com and torproject.org,
are blocked by DNS and HTTP inspection.
Blocked requests are redirected to a block page.
\item{DNS-only censor.}
This censor injects false replies to DNS queries
for forbidden domain names (DNS poisoning),
but does not do deep packet inspection on TCP streams.
Unlike the corporate censor, this censor allows protocols
other than HTTP, HTTPS, and DNS.
\item{Comprehensive censor.}
Employs a variety of techniques, depending on the target.
May do DNS poisoning, IP blocking, and inspection of TCP streams
(examining the URL of HTTP requests, for example).
Some domains may be blocked by DNS and deep packet inspection;
others may additionally be blocked by IP address.
Tor relays are blocked by IP address.
Some domains, like wikipedia.org, may allow HTTP but not HTTPS.
Blocked requests fail silently (no block page).
\end{itemize}

We will divide our participants into treatments
by censorship environment.
All participants will be asked to complete the same
list of tasks.
We constructed the list by looking at some of the top
Alexa sites and imagining what might be reasonable
and relevant tasks for each
(that do not require a participant to reveal private information
like passwords).
The tasks are not all intended to be challenging to complete.
Some may not even require the use of Tor Browser,
depending on the censorship environment in effect.

\begin{itemize} \itemsep1pt \parskip0pt \parsep0pt
\item Do a Google web search.
\item Watch a YouTube video.
\item Find the Amazon best-selling books.
\item Find Yahoo's exchange rate between dollars and euros.
\item Read the Wikipedia featured article.
\item Find the Twitter trending topics.
\item Find directions on Bing Maps.
\end{itemize}

\section{What we want feedback on}

\noindent {\bfseries 1. testing the pluggable transports} blah explain ourselves. 

\noindent {\bfseries 2. censorship environments} blah explain ourselves. 

\noindent {\bfseries 3. list of tasks} blah explain ourselves. 

\noindent {\bfseries 4. general design feedback} (size, recruitment pool --ie general or journalists) blah explain ourselves. 

\section{Resources}

\url{https://blog.torproject.org/blog/ux-sprint-2015-wrapup}

\url{https://trac.torproject.org/projects/tor/wiki/org/meetings/2015UXsprint}

\url{https://people.torproject.org/~dcf/uxsprint2015/}

\bibliographystyle{abbrv}
\bibliography{bibliography.bib} 
\end{document}
