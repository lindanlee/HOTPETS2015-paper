\documentclass{sig-alternate-hotpets15}
\usepackage{color}
\usepackage[hyphens]{url}
\usepackage{longtable}
\usepackage{graphicx}
\usepackage{enumitem}
\usepackage{pdfpages}

\def\etal{{\it et al.~}}
\newenvironment{packed_enum}{
\begin{enumerate}
  \setlength{\itemsep}{1pt}
  \setlength{\parskip}{0pt}
  \setlength{\parsep}{0pt}
}{\end{enumerate}}
\newenvironment{packed_item}{
\begin{itemize}
  \setlength{\itemsep}{1pt}
  \setlength{\parskip}{0pt}
  \setlength{\parsep}{0pt}
}{\end{itemize}}

\begin{document}

\title{Assessing Usability of Tor Browser}
\numberofauthors{1}
\author{
 \alignauthor David Fifield\textsuperscript{1} and Linda N. Lee\textsuperscript{1}, Serge Egelman\textsuperscript{1,2}, David Wagner\textsuperscript{1}\\
   \vspace{0.5em}
   \affaddr{\textsuperscript{1}University of California, Berkeley, \{fifield,lnl,egelman,daw\}@cs.berkeley.edu}, \\
   \affaddr{\textsuperscript{2}International Computer Science Institute, egelman@icsi.berkeley.edu}\\
}

%\numberofauthors{1}
%\author{
% \alignauthor Anonymous\\
%   \vspace{0.5em}
%   \affaddr{Some Place}
%}
\maketitle

\begin{abstract}
Our talk will be about ongoing experiments aimed at
assessing and improving the usability of Tor Browser.
There have been been no major usability evaluations of
Tor Browser since the introduction of the 4.0 series,
which introduced radical UI changes.
We will briefly describe the results of a completed study
that examined downloading and installing the browser
and using it for basic browsing tasks.
This study uncovered a number of bugs and stopping points
and has already effected concrete change in the browser.
We will describe our plans for future studies aimed at
isolating specific browser features in depth,
with a detailed design of an experiment to evaluate
Tor Browser as a tool for censorship circumvention.
We solicit feedback on our experimental design
and the specific features of the browser that we are testing.
\end{abstract}

\keywords{Privacy, Security, User Studies, Anonymity, Tor} % NOT required for Proceedings

\section{What we already did/background stuff}
Tor is an anonymity tool which is used by a lot of people. It is important (prove this/cite stuff/say something more meaningful about this) and works well. However, there are complaints that Tor is not usable. Norcie did an experiment \cite{norcie2012eliminating} which identified stopping points in Tor browser, finding out when people would get frustrated with Tor so much they would stop using it. But what about the people who are using Tor? what usability issues would thee be if people were not stopped? 

To our knowledge, that has been the only user study of Tor. Since then, Tor has a lot of updates. Additionally, there were a lot of features that were untested. We ran a small pilot study on 5 journalists which involved a cognitive walkthrough.(explain cognitive walkthrough if necessary). We did this to find new stopping points, but also to see how they would download Tor, if they could understand the address bar/menu, how well they were able to complete basic tasks (searching, setting new circuits, etc), and generally if they had any usability complaints about the browser. 

Recruitment here. Demographics here. Setup here. Explain that other Tor users could see this as well. How long each the experiment took on average here. (put any misc details or additional methodology things here). The result was that people did have difficulty with X, Y, Z.

\section{Design}
\#STORYTIMEE  about why we chose this experiment. 

Design of experiment. 

Methodology of experiment. 

What we hope to get from this experiment. 

\section{What we want feedback on}

\noindent {\bfseries 1. testing the pluggable transports} blah explain ourselves. 

\noindent {\bfseries 2. censorship environments} blah explain ourselves. 

\noindent {\bfseries 3. list of tasks} blah explain ourselves. 

\noindent {\bfseries 4. general design feedback} (size, recruitment pool --ie general or journalists) blah explain ourselves. 


\bibliographystyle{abbrv}
\bibliography{bibliography.bib} 
\end{document}
