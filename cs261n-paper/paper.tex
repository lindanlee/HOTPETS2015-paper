\documentclass{sig-alternate-hotpets15}
\usepackage{color}
\usepackage[hyphens]{url}
\usepackage{longtable}
\usepackage{graphicx}
\usepackage{enumitem}
\usepackage{pdfpages}

\def\etal{{\it et al.~}}
\newenvironment{packed_enum}{
\begin{enumerate}
  \setlength{\itemsep}{1pt}
  \setlength{\parskip}{0pt}
  \setlength{\parsep}{0pt}
}{\end{enumerate}}
\newenvironment{packed_item}{
\begin{itemize}
  \setlength{\itemsep}{1pt}
  \setlength{\parskip}{0pt}
  \setlength{\parsep}{0pt}
}{\end{itemize}}

\begin{document}

\title{better title eventually}
\numberofauthors{1}
\author{
 \alignauthor Linda N. Lee, David Fifield\\
   \affaddr{University of California, Berkeley, \{lnl, fifield\}@cs.berkeley.edu} \\
   \vspace{0.5em}
   CS261N Final Report, \today
}
\maketitle

% is the abstract missing some oomph? 
% I don't know how much the storyline is showing through.. or what we are testing. 
\begin{abstract}
It is easy to drastically overestimate user comprehension 
and therefore be misguided for user behavior. 
Even though people download and install things all the time, 
they do not know how to do it securely, and worse, 
they are habituated into doing it a certain way. 
What users had trouble with were the most basic tasks, 
such as downloading and installing Tor, rather than 
nagivating a robust menu or understanding subtleties of security guarentees. 
This is really troubling, and matters for things like Tor. 
We use this insight to fix a bunch of stuff and also 
present a new experiment with a basic enough focus, 
making us the first to focus on censorship too. 
\end{abstract}

\keywords{User Studies, Tor, Anonymity, Censorship, Security}

\section{Introduction} %this should CLEARLY state the results. 
Talk about the results and contributions concretely.
As short as possible summary of UX sprint here. %only what is necessary to understand results. 
Talk about the changes resulted in the Tor browser here.
As short as possible summary of lessons learned, 
and also a short as possible motivation for censorship work here. %yep, same.
Talk about the experiment design quickly here. 

\section{Background}  %ramp people up! 

\subsection{Tor} 
Talk about anything relevant to Tor. Just enough background so
that people who have never heard of it will know what I am talking
about, but no more than that. 

\subsection{User behavior}
Then also talk about any of the behaviors we are trying to mitigate,
or talk about the average user and what they know, how they are motivated,
just sketching out the environment in general. 

\subsection{Usable security impact} 
cite work which has made a huge impact on how people use it, and 
other things which still remain broken because of usability issues. 

We don't want Tor to be like this, and want people to use it!

\subsection{Censorship} 
Motivational paragraph time about how this is so everywhere. 

Talk about the basic methods of censoring, and how it differs 
around the world, which sites are targeted, etc. 

Scope the paper here. Talk about how we are not up Cuba-like adversaries. 

\section{UX sprint} 
Details. Gory gory details of how it all went down. 
Include the recruitment process, the setup, what the exact 
tasks were for participants, demographics, everything. 
It all should have a home here. 

\section{Results}
Present situations which really were unexpected. 
Also extract the golden quotes and state them here 
to prove points I want to make.  

\subsection{Impact}
whoo talk about the cool stuff that happened next here. 
This means talking about the tickets and changes made 
to the Tor browser, the start of a user manual, and 
increased Tor UX efforts as a whole.

\subsection{Takeaways} 
highlight what we learned and use this to transition into 
the design of the next experiment. 

Wanted reproducibility of experiment.
Wanted more people for a representative population. 
Also make practical considerations such as avoiding transcription. 

\section{Experiment Design} 
an informed, user-conscientious, user-focused experiment design to target 
narrow sets of basic features which are used for censorship. 

\subsection{Methodology}  
Talk about the justification of the setup and the general 
work flow of an experiment session. Address any details up front
about computers, money, people, etc. 

\subsection{Censorship Environments} 
Basically a section like the HOTPETS paper section.
I should add details of exact configurations if available. 

\subsection{Tasks}
A section like the HOTPETS paper section. 
I  should add details of exact tasks if available.

\subsection{Feedback} 
Talk about the exit survey and what that covers. 
Explain why we want to know what we are asking about. 
And why this matters. 

\section{Discussion} 
Discuss the hypotheses we have here for the experiment. 

And any work that needs to be done for the experiment. 

Next stps to take after the experiment, can also be here. %is this too far away? 

\section{Related Work} %this is distracting to have earlier. %don't overdo the citations. 
The stopping points paper on Tor's usability ~\cite{norcie2012eliminating}. 
And that is pretty much it for Tor usability stuff. %find more if it exists

I guess that other literature on general usability applies here too. %so search for it. 

\section{Conclusion} 
Do not repeat the abstract, introduction, or the results section. 
Focus this time to discuss the limitations of the work and broader impacts. 

Reach out to the community to make a call for more a focus on usability for security products. 


\bibliographystyle{abbrv}
\bibliography{bibliography.bib} 
\end{document}
