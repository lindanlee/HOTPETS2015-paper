%\documentclass{article}
\documentclass[letterpaper,twocolumn,10pt]{article}
\usepackage[margin=0.75in]{geometry}
\usepackage{color}
\usepackage[hyphens]{url}
\usepackage{longtable}
\usepackage{graphicx}
\usepackage{enumitem}
\usepackage{pdfpages}

\def\etal{{\it et al.~}}
\newenvironment{packed_enum}{
\begin{enumerate}
  \setlength{\itemsep}{1pt}
  \setlength{\parskip}{0pt}
  \setlength{\parsep}{0pt}
}{\end{enumerate}}
\newenvironment{packed_item}{
\begin{itemize}
  \setlength{\itemsep}{1pt}
  \setlength{\parskip}{0pt}
  \setlength{\parsep}{0pt}
}{\end{itemize}}

\begin{document}

\title{better title eventually}
\author{Linda N. Lee, David Fifield\\
University of California, Berkeley, \{lnl, fifield\}@cs.berkeley.edu
}
\maketitle

% is the abstract missing some oomph? 
% I don't know how much the storyline is showing through.. or what we are testing. 
\begin{abstract}
\indent \indent It is easy to drastically overestimate user comprehension 
and therefore be misguided for user behavior. 
Even though people download and install things all the time, 
they do not know how to do it securely, and worse, 
they are habituated into doing it a certain way. 
What users had trouble with were the most basic tasks, 
such as downloading and installing Tor, rather than 
nagivating a robust menu or understanding subtleties of security guarentees. 
This is really troubling, and matters for things like Tor. 
We use this insight to fix a bunch of stuff and also 
present a new experiment with a basic enough focus, 
making us the first to focus on censorship too. 
\end{abstract}

\section{Introduction} %this should CLEARLY state the results. 
\indent \indent Talk about the results and contributions concretely.
As short as possible summary of UX sprint here. %only what is necessary to understand results. 
Talk about the changes resulted in the Tor browser here.
As short as possible summary of lessons learned, 
and also a short as possible motivation for censorship work here. %yep, same.
Talk about the experiment design quickly here. 

\section{Background}  %ramp people up! 

\subsection{Tor} 
\indent \indent Talk about anything relevant to Tor. Just enough background so
that people who have never heard of it will know what I am talking
about, but no more than that. 

\subsection{User behavior}
\indent \indent Then also talk about any of the behaviors we are trying to mitigate,
or talk about the average user and what they know, how they are motivated,
just sketching out the environment in general. 

\subsection{Usable security impact} 
\indent \indent cite work which has made a huge impact on how people use it, and 
other things which still remain broken because of usability issues. 

We don't want Tor to be like this, and want people to use it!

\subsection{Censorship} 
\indent \indent Motivational paragraph time about how this is so everywhere. 

Talk about the basic methods of censoring, and how it differs 
around the world, which sites are targeted, etc. 

Scope the paper here. Talk about how we are not up Cuba-like adversaries. 

\section{UX sprint} 
\indent \indent Talk about it just enough that it makes a good storyline for the experiment that we 
are proposing a design of. Motivate it a little bit, and tell a story. Then refer to the 
Appendix for additional information.  

\subsection{Impact}
\indent \indent whoo talk about the cool stuff that happened next here. 
This means talking about the tickets and changes made 
to the Tor browser, the start of a user manual, and 
increased Tor UX efforts as a whole.

\subsection{Takeaways} 
\indent \indent highlight what we learned and use this to transition into 
the design of the next experiment. 

Wanted reproducibility of experiment.
Wanted more people for a representative population. 
Also make practical considerations such as avoiding transcription. 

\section{Experiment Design}

\subsection{WHAT IS THIS}

\noindent {\bfseries Motivation and Background Story} talk about the generalizability of the work, what we aim to find, and what future work this will lead to. Be sure to sketch the storyline of honing in on specific features. 

Tor has grown beyond its original purpose as and has 
since become an important Internet circumvention tool.
We specifically examine it usability as a censorship circumvention tool,
an essential facet for adoption and use.  
We focus our analysis on the connection configuration dialog of Tor browser,
as censorship circumvention requires correct transport configurations. %%
This study is aimed at evaluating 
if and how easily users can circumvent censorship using Tor Browser,
isolating specific browser features to study in the process. 
To this end, we will conduct a large-scale user study examining hundreds of users 
on how they navigate Tor's configuration wizard to complete seven browsing tasks 
in three different adversarial settings. 

\subsection{Overview}
\indent \indent For users to circumvent censorship in their resident countries, they will need to configure Tor to set up a proxy, bridge, or both. There is a configuration wizard to help guide users through setting up their connection, but our hypothesis is that the average user will not easily be able to configure Tor correctly in these adverse settings, as they require the user to provide IP addresses of proxies or bridges to use. The goal of our experiment is to see how successful users are at carrying out common browsing tasks in an adversarial setting using Tor. 

We start off the experiment by telling them that they are in an adversarial setting, and that some websites are blocked and some websites are not. We will instruct them to visit notblocked.com and also blocked.com to illustrate the situation that they are in. We will also explain what Tor is, and how they can use it if necessary, and ask them to complete the set of tasks. To complete all the tasks, the participants will ultimately need to get the correct configuration settings for the country they are simulated to be in. We will end the experiment with a survey which asks them things like their security background, tech exposure, what was unexpected, what was hard, etc. We plan to analyze: which configuration tasks are hard (bridges versus proxies, etc.), how many were successful, how long it took users to configure, if they understood the process, if they were confident that it was working as expected, etc. 

\subsection{Censorship Environment Simulation}
\indent \indent For our experiment, we plan to simulate and test three distinct censorship environments
which vary in their methods of censorship and thusly require distinct responses from our 
participants to circumvent successfully. These environments are designed with respect to 
common censorship techniques employed 
today along with knowledge of how pluggable transports will need to be used.
For instance, some of Tor's pluggable transports work in censorship environments
where others do not. Additionally, some of them require additional information (like bridge 
addresses) before they enable the user to successfully circumvent censorship. 
Details of the three censorship environments are below, increasing in 
order of censorship comprehensiveness and, therefore, difficulty to circumvent. 

These environments are not meant perfectly to replicate the network environment
in any particular country.  Censorship environments in are complex and volatile, making
the task of replicating a network environment infeasible. More importantly, replicating these
environments is not critical to this experiment. What is required for this experiment are 
environments which would require users to complete similar Tor configuration 
tasks before they are able to circumvent the censorship environment. We set up the following 
abstract simulations which are inspired by reality and require users to complete various tasks
to circumvent censorship. 

\begin{itemize} \itemsep1pt \parskip0pt \parsep0pt
\item{\bfseries Corporate network.}
A simulation of an enterprise or educational firewall.
Blocks all services but HTTP, HTTPS, and DNS.
Certain non-work-related domains, like youtube.com and torproject.org,
are blocked by DNS and HTTP inspection.
Blocked requests are redirected to a block page.
%what do users need to do to circumvent this? 
\item{\bfseries DNS-only censor.}
This censor injects false replies to DNS queries
for forbidden domain names (DNS poisoning),
but does not do deep packet inspection on TCP streams.
Unlike the corporate censor, this censor allows protocols
outside a small whitelisted set.
%what do users need to do to circumvent this? 
\item{\bfseries Comprehensive censor.}
Employs a variety of techniques, depending on the target.
May do DNS poisoning, IP blocking, and inspection of TCP streams
(examining the URL of HTTP requests, for example).
Some domains may be blocked by DNS and deep packet inspection;
others may additionally be blocked by IP address.
Tor relays are blocked by IP address.
Some domains, like wikipedia.org, may allow HTTP but not HTTPS.
Blocked requests fail silently (no block page).
%what do users need to do to circumvent this? 
\end{itemize}

We plant to present our work at HotPETS2015 and solicit feedback on the design 
of this experiment. Specifically, we would want a reliable way to target the testing of 
specific transports. But given that the average user would not have domain knowledge of 
what transports are for Tor, is there anything to infer from participants' selection of transports, 
or do we assume they are trying transports at random?

\subsection{Web Browsing Tasks}
\indent \indent The tasks below are designed to be easy and quick to complete for an average
computer user. This design is critical since the objective of the experiment is to observe
how comfortable users are with circumventing censorship, which we indirectly measure
by taking data regarding how participants complete the task. Giving participants tasks 
they are not comfortable with performing or sending them to websites which they are 
unfamiliar with will introduce more variables into the experiment to account for. 
Specifically, we would need to separate any confusion caused by the task itself versus
the task of circumventing censorship. 

To create a set of tasks that an average user will be able to complete reliably, 
we turned to the top Alexa sites, the most popular websites on the Internet. This 
is an indication of representative and relevant browsing behavior. To reach the final
set of website destinations, we selected sites which were popular, but also commonly
censored around the world to make our study more representative. The tasks 
associated with each website were crafted with user familiarity in mind but narrowed 
down the possibilities by removing tasks which caused ethical concerns (such as 
requiring a user to log in to a websites, which would reveal login information).  
We believe that these tasks we will use for our study will be simple tasks for an average 
user to complete, on familiar websites. 

All participants will be given the same set of tasks, regardless of their simulated
censorship environment.  Although every participant will attempt to complete the same 
set of tasks, the difficulty of completing these tasks will vary based on their chosen environment. 
Depending on the censorship environment, some tasks may not even require the use or Tor Browser. 

\begin{itemize} \itemsep1pt \parskip0pt \parsep0pt
\item Do a Google web search.
\item Watch a YouTube video.
\item Find the Amazon best-selling books.
\item Find Yahoo's exchange rate between dollars and euros.
\item Read the Wikipedia featured article.
\item Find the Twitter trending topics.
\item Find directions on Bing Maps.
\end{itemize}

We plan to present our work at HotPETS2015 and solicit feedback on the design
of this experiment. Circumvention is broader than the stereotypical ``dissident blogger'' 
use case. We will solicit feedback on which additional tasks will provide additional insight,
or if there are different tasks which are representative of other censorship evasion use cases.

\section{Methodology}  

\indent \indent Talk about the two-pronged qualitative and empirical data measurement stuff here. 

\subsection{Qualitative Data} 
Write about the exit survey here. Write down number of questions, surveygizmo, etc.

\subsection{Empirical Data} 
Screen capture, timing, completion, clicks, etc. Things to measure about the activity. 

\subsection{Experimental Setup} 

\indent \indent We plan to recruit up to 200 users for the purpose of this study, making this the largest user 
study to date examining Tor. We do this in order to increase our chances at recruiting a diverse
user population, ranging in age, gender, technical fluency, and other characteristics. Since Tor 
does not keep information regarding its user base, we will try to obtain a user base representative of
the average Internet using-population. Additionally, recruiting a large number of participants has the 
benefit of ensuring enough samples in order to have significant effect sizes in 
each simulated censorship environment. 

This study will be conducted at the Experimental Social Science Laboratory (Xlab)
at the University of California, Berkeley. Recruitment will be conducted through Craigslist. 
Although Xlab provides a portal in which researchers can recruit a pre-registered user base,
we choose to not recruit through this portal, as their participants mostly consist of mostly Berkeley 
students. The laboratory setting consists of 36 Toshiba Tecra R850 laptops with cubicle walls 
separating each laptop. Since we do not have the ability to employ network-level firewalls in this
laboratory setting, we will use individual host firewalls to simulate the censorship environments.  
We will write and employ a browser extension will record user activity, such as how many times they 
clicked, what websites they were successfully able to visit, and how long they took to complete 
each task. This is because WHY? %ask david exactly why--technically. isn't the first hop in the clear? 

The total length of the experiment, including briefing, completing the censorship circumvention tasks, 
exit survey, and debriefing, will take less than an hour. Participants will be compensated \$30 in the form
of a check or Visa debit card for their time, which is enough to cover minimum wage during their participation 
and their transportation costs to and from the laboratory. 

\section{Discussion} 

\subsection{Limitations}
\indent \indent One limitation of our experiment is that participants' behaviors in a laboratory setting may
be different to participants' behaviors in the wild. Studies have shown that participants will act in a way which 
will aim to please researchers, trying to simulate what they believe are the results that we desire. Our participants
may be motivated to complete the task in order to succeed at the experiment. One possible effect is that a user
may have more patience to try to succeed at configuring the Tor Browser correctly during the experiment 
compared to if they were attempting the same task at home. 

Another limitation of our experiment is that our participants might not have true interest or motivation in
circumventing censorship. We could instead choose to recruit only users who have actively circumvented
censorship or have an interest in Tor, but that study would also come with its own biases and limitations.
We believed that both are worthwhile endeavors; however, we chose to aim for the general population. 
This is because we would eventually want Tor's configuration dialog to be usable for a wide array of users, 
including non-power users, and other types users who might not yet use Tor. 

\section{Future Work}
And any work that needs to be done for the experiment. Give a brief timeline.

Next stps to take after the experiment, can also be here. %is this too far away? 

\section{Related Work} %this is distracting to have earlier. %don't overdo the citations. 
The stopping points paper on Tor's usability ~\cite{norcie2012eliminating}. 
And that is pretty much it for Tor usability stuff. %find more if it exists

I guess that other literature on general usability applies here too. %so search for it. 
Especially look into browser configuration work. %do it to it. 

\section{Conclusion} 
Do not repeat the abstract, introduction, or the results section. 
Focus this time to discuss the limitations of the work and broader impacts. 

Reach out to the community to make a call for more a focus on usability for security products. 

\appendix

\section{UX sprint} 
Details. Gory gory details of how it all went down. 
Include the recruitment process, the setup, what the exact 
tasks were for participants, demographics, everything. 
It all should have a home here. 

\subsection{Motivation}
Talk about why we did what we did, and what the end goal 
of the experiment was. Also talk about how we expected users
to be much better at installing and downloading than they were. 

\subsection{Experiment Overview}
Talk about how the experiment was for an average user. We start
with the cognitive walkthrough, then have them do one on a dummy
and calibration task, we give feedback on the walkthrough, and then 
give them the freedom to complete the web tasks. 

GET THE LIST OF TASKS WE MADE THEM DO. 

\subsection{Methodology}
How we recruited, what the setup was like, how we got the tasks, details of 
how the experiment was run, and things like that. 

\subsection{Results}
Present situations which really were unexpected. 
Also extract the golden quotes and state them here 
to prove points I want to make.  

\url{https://blog.torproject.org/blog/ux-sprint-2015-wrapup}

\subsection{Impact}
whoo talk about the cool stuff that happened next here. 
This means talking about the tickets and changes made 
to the Tor browser, the start of a user manual, and 
increased Tor UX efforts as a whole.

\subsection{Resources}

Our online artifacts of our completed pilot study, such as 
the summary, results, and resulting browser changes are below:
\begin{itemize} \itemsep1pt \parskip0pt \parsep0pt
\item \url{https://trac.torproject.org/projects/tor/wiki/org/meetings/2015UXsprint}
\item \url{https://blog.torproject.org/blog/ux-sprint-2015-wrapup}
\end{itemize}

\noindent Subtitled screen videos:
\begin{itemize} \itemsep1pt \parskip0pt \parsep0pt
\item \url{https://people.torproject.org/~dcf/uxsprint2015/}
\end{itemize}

\bibliographystyle{abbrv}
\bibliography{bibliography.bib} 
\end{document}
